\section{Signals and basics}

\question{In digital Communication system which type of waveforms are propagated in the channel?}

\question{Why do we need a practical definition of bandwidth like the '3dB bandwidth'?}
\begin{checkboxes}
    \choice Because finite duration signals have finite support in the frequency domain.
    \CorrectChoice Because finite duration signals have infinite support in the frequency domain.
    \choice It is just a common practice. Defining the bandwidth as the support of the signal in the frequency domain works as well for practical applications.
    \choice Because it is easier to measure a 3dB bandwidth rather than other definitions of bandwidth.
    \choice None of the other options.
\end{checkboxes}

\question{Which of the following statements about PSK modulation is TRUE?}
\begin{checkboxes}
    \choice Since each symbol has a different energy level, operating close to the saturation region of a power amplifier does not cause distortion, resulting in efficient use of the amplifier.
    \choice Since each symbol has a different energy level, operating close to the saturation region of a power amplifier causes distortion, resulting in a larger probability of error.
    \choice Since all the symbols have the same energy, operating close to the saturation region of a power amplifier causes distortion, resulting in a larger probability of error.
    \CorrectChoice Since all the symbols have the same energy, operating close to the saturation region of a power amplifier does not cause distortion, resulting in efficient use of the amplifier.
    \choice None of the other options.
\end{checkboxes}

\question{Put the building blocks of a digital communication system in the correct order:}
\begin{solution}
    TX $\,\to\,$ Encoder $\,\to\,$ Modulator  $\,\to\,$ Channel  $\,\to\,$ Demodulator  $\,\to\,$ Decoder  $\,\to\,$ RX
\end{solution}

\question{A digital communication system uses a 2-PAM modulation with rectangular pulses and a given average energy per symbol $E_s$. What happens if we adopt instead a 4-PAM modulation, using the same basic pulse and average energy per symbol $E_s$?}
\begin{checkboxes}
    \choice The bitrate decreases, but the system becomes more robust to errors.
    \CorrectChoice The bitrate increases, but the system becomes more error prone.
    \choice The bandwidth efficiency increases resulting in a generally larger bandwidth of the transmitted signals.
    \choice The bandwidth efficiency decreases resulting in a generally larger bandwidth of the transmitted signals.
    \choice None of the other options.
\end{checkboxes}

\question{Which operation performed on a signal allows moving its frequency content around a desired frequency?}
\begin{checkboxes}
    \CorrectChoice Multiplication with a sinusoidal function
    \choice Convolution with a sinusoidal function
    \choice Amplitude modulation
    \choice Lowpass filtering
    \choice None of the other options.
\end{checkboxes}

\question{In a digital modulation scheme transmitting  $M$ different symbols, what could be a wise strategy to map each sequence of $\log_2 M$ bits to each symbol?}
\begin{checkboxes}
    \choice It does not matter. As long as each symbol carries the same number of bits, their mapping has no impact on the performance.
    \choice Maximizing the number of bits that differ between symbols that are more likely to be mistaken is a good strategy to minimize the bit error rate.
    \CorrectChoice Minimizing the number of bits that differ between symbols that are more likely to be mistaken is a good strategy to minimize the bit error rate.
    \choice The bit sequence with the largest number of '1's should be assigned to the symbol with the highest energy, and so on, until the sequence with all zeros is assigned to the symbol with the lowest energy.
    \choice None of the other options.
\end{checkboxes}
