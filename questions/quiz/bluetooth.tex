\section{Bluetooth protocol and security}

\question{How does Bluetooth Classic handle privacy concerns compared to Bluetooth LE?}
\begin{checkboxes}
    \choice Bluetooth Classic uses encryption keys for all data transmissions.
    \choice Bluetooth Classic limits the number of devices that can connect simultaneously.
    \choice Bluetooth Classic randomizes MAC addresses for improved privacy.
    \choice Bluetooth Classic does not have any privacy features.
    \choice None of the other options.
\end{checkboxes}

\question{What is the Bluesnarfing attack?}
\begin{checkboxes}
    \choice A vulnerability that allows unauthorized access to Bluetooth devices via malicious code execution.
    \choice A method for securely pairing Bluetooth devices using encryption keys.
    \choice A feature that enhances Bluetooth data transmission speed.
    \choice A protocol used for secure pairing in Bluetooth devices.
    \CorrectChoice None of the other options.
\end{checkboxes}

\question{Which security services does Bluetooth support?}
\begin{checkboxes}
    \CorrectChoice Authorization
    \CorrectChoice Confidentiality
    \CorrectChoice Authentication
    \choice Notarization
    \choice None of the other options.
\end{checkboxes}


\question{How prevent MITM on Bluetooth Secure Simple Pairing?}

\question{Which multiple access mechanism BT uses?}

\question{What is Bluetooth LE Privacy Feature?}
\begin{checkboxes}
    \choice A feature that encrypts all data transmitted over Bluetooth Low Energy (LE).
    \CorrectChoice A feature that generates random MAC addresses for advertising packets.
    \choice A feature that hides the Bluetooth device from unauthorized scanning.
    \choice A feature that restricts the range of Bluetooth connections to improve privacy.
    \choice None of the other options.
\end{checkboxes}


\question{What is pairing in Bluetooth technology? Select one:}
\begin{checkboxes}
    \choice A method for managing power consumption in Bluetooth devices.
    \CorrectChoice The process of establishing a Bluetooth connection between two devices.
    \choice A mechanism for securely storing Bluetooth device information for future connections.
    \choice None of the other options.
    \choice A protocol for encrypting Bluetooth data transmissions.
\end{checkboxes}


\question{Consider the Bluetooth Authentication sketched in the figure below.}
\begin{solution}
    \begin{enumerate}
        \item A sends its Public Key PKa
        \item B sends its Public Ket PKb
        \item B selects a Random value Nb
        \item A selects a Random value Na
        \item B computes the confirmation Cb = f4(Pkb, Pia, Nb, 0)
        \item B sends the confirmation Cb to A
        \item A sends its nonce Na
        \item B sends its nonce Nb
        \item A checks if Cb = f4(Pkb, Pia, Nb, 0)
    \end{enumerate}
\end{solution}

\question{What is bonding in Bluetooth technology?}
\begin{checkboxes}
    \choice A mechanism for optimizing Bluetooth connection range.
    \choice A method for encrypting data packets during Bluetooth transmission.
    \CorrectChoice A process of permanently storing Bluetooth device information after successful pairing.
    \choice A protocol for managing multiple simultaneous Bluetooth connections.
    \choice None of the other options.
\end{checkboxes}

\question{What is the Relay Attack in Bluetooth?}
\begin{checkboxes}
    \CorrectChoice It bridges communications between devices to fool them into thinking they are close to each other
    \choice It consists in brute forcing the PIN.
    \choice It sends a previously recorded frame to misuse the communication.
    \choice It leverages the signal to noise ratio to authenticate the device.
    \choice None of the other options
\end{checkboxes}
