\subsection{Basic Network Security Concepts and Cryptography}

\question{Which of the following best defines a stream cipher?}
\begin{checkboxes}
    \CorrectChoice A cipher that encrypts data one bit or byte at a time, using a key stream generated from a key.
    \choice A cipher that rearranges the order of letters in the plaintext.
    \choice A cipher that substitutes each letter of the plaintext with multiple different letters or symbols based on a complex substitution scheme.
    \choice A cipher that encrypts the plaintext into binary code.
    \choice None of the other options.
\end{checkboxes}

\question{Which of the following best defines a block cipher?}
\begin{checkboxes}
    % Source: google doc
    \choice A cipher that rearranges the order of letters in the plaintext.
    \choice A cipher that substitutes each letter of the plaintext with multiple different letters or symbols based on a complex substitution scheme.
    \choice A cipher that encrypts data one bit or byte at a time, using a key stream generated from a key.
    \choice A cipher that encrypts the plaintext into binary code.
    \CorrectChoice None of the other options.
\end{checkboxes}

\question{Which of the following defines a cipher text attack?}

\question{what is known plaintext attack}

\question{Classification of wireless network, adhoc/infrastructure, fixed/mobile, wwan/wlan}

\question{Which of the following best defines asymmetric encryption?}
\begin{checkboxes}
    \choice A cipher that encrypts data in fixed-size blocks, each block processed independently with a key.
    \choice A cipher that uses a single key for both encryption and decryption.
    \CorrectChoice A cipher that uses a pair of keys (public and private) for encryption and decryption to ensure secure communication between parties.
    \CorrectChoice A cipher that uses different keys for encryption and decryption, ensuring secure communication between parties.
    \choice None of the other options.
\end{checkboxes}


\question{Associate the correct definition to the following security mechanisms}
\begin{solution}
    \begin{itemize}
        \item \textbf{Access Control}: Determines and enforce the access rights of the entity depending on the authenticated identity. Provides confidentiality for either data or traffic flow information.
        \item \textbf{Encipherment}: Provides confidentiality for either data or traffic flow information.
        \item \textbf{Traffic Padding}: Protects against traffic analysis attacks by adding non essential data to network communications. Assures data integrity, origin, time, and destination about data communicated between two or more entities.
        \item \textbf{Notarization}: Assures data integrity, origin, time, and destination about data communicated between two or more entities.
    \end{itemize}
\end{solution}

\question{Associate the correct definition of the following Security Services:}
\begin{solution}
    \begin{itemize}
        \item \textbf{Non-repudiation:} Assurance that someone cannot deny the validity of something.
        \item \textbf{Confidentiality:} Ensuring that information is accessible only to those authorized to have access.
        \item \textbf{Availability:} Ensuring that authorized users have access to information and associated assets when required.
        \item \textbf{Integrity:} Maintaining and assuring the accuracy and completeness of data over its entire lifecycle.
    \end{itemize}
\end{solution}

\question{What is the key difference between a monoalphabetic and a polyalphabetic cipher?}
\begin{checkboxes}
    \choice The method of rearranging the order of letters.
    \choice The use of multiple keys for encryption.
    \choice The length of the plaintext that can be encrypted.
    \CorrectChoice The complexity of the substitution pattern used.
    \choice None of the other options.
\end{checkboxes}

\question{Which of the following best defines a ciphertext-only attack in the context of wireless network security?}
\begin{checkboxes}
    \CorrectChoice An attack where the attacker can only access the ciphertext and attempts to decrypt it without any additional information.
    \choice An attack where the attacker has access to both the plaintext and its corresponding ciphertext.
    \choice An attack where the attacker manipulates the ciphertext to produce a predictable change in the plaintext.
    \choice An attack where the attacker intercepts and alters the communication between two parties.
    \choice None of the other options.
\end{checkboxes}

\question{Which of the following best defines a chosen-plaintext attack in the context of wireless network security?}
\begin{checkboxes}
    \CorrectChoice None of the other options.
    \choice An attack where the attacker can only access the ciphertext and attempts to decrypt it without any additional information.
    \choice An attack where the attacker has access to both the plaintext and its corresponding ciphertext.
    \choice An attack where the attacker manipulates the ciphertext to produce a predictable change in the plaintext.
    \choice An attack where the attacker intercepts and alters the communication between two parties.
\end{checkboxes}

\question{Which of the following definitions correctly defines a \textbf{Passive Attack} in a wireless network?}
\begin{checkboxes}
    \choice An attack that involves injecting malicious packets into the network.
    \choice An attack where the attacker modifies or disrupts the communication.
    \choice An attack where the attacker gains unauthorized access to the network resources.
    \choice An attack where the attacker intercepts and monitors the wireless communication, altering the data.
    \CorrectChoice None of the other options
\end{checkboxes}

\question{Which of the following are security services?}
\begin{checkboxes}
    \choice Encryption
    \choice Password Checking
    \choice Message Authentication Code
    \CorrectChoice Confidentiality
    \choice None of the others
\end{checkboxes}

\question{In a Wireless Personal Area Networks (WPAN), which of the following configurations are possible?}
\begin{checkboxes}
    % Source: google doc
    \CorrectChoice Master-slave mode
    \CorrectChoice Mesh mode
    \choice Radio Access Network and Core Transport Network
    \choice Point to point network
    \choice None of the others
\end{checkboxes}

\question{Which of the following best defines the Access Control service in the context of wireless networks?}
\begin{checkboxes}
    % source: google doc
    \CorrectChoice A service that restricts access to network resources based on user identity and predefined policies.
    \choice A service that monitors and logs network traffic for potential threats and attacks.
    \choice A service that ensures data integrity by verifying that data has not been altered during transmission.
    \choice A service that provides encryption to secure data transmitted over the network.
    \choice None of the other options.
\end{checkboxes}

\question{Which of the following best defines traffic padding in the context of wireless network security?}
\begin{checkboxes}
    \choice A service that verifies the identity of users and devices on the network.
    \choice A method for ensuring data integrity during transmission.
    \choice A process of compressing data to reduce transmission time.
    \choice A method used to encrypt data to prevent unauthorized access.
    \CorrectChoice None of the other options.
\end{checkboxes}
