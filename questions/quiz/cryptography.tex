\section{Cryptography}

\question{What is a stream cipher?}

\question{Which of the following defines a cipher text attack?}

\question{what is known plaintext attack}

\question{Classification of wireless network, adhoc/infrastructure, fixed/mobile, wwan/wlan}

\question{Which of the following best defines asymmetric encryption?}
\begin{checkboxes}
    \choice A cipher that encrypts data in fixed-size blocks, each block processed independently with a key.
    \choice A cipher that uses a single key for both encryption and decryption.
    \CorrectChoice A cipher that uses a pair of keys (public and private) for encryption and decryption to ensure secure communication between parties.
    \CorrectChoice A cipher that uses different keys for encryption and decryption, ensuring secure communication between parties.
    \choice None of the other options.
\end{checkboxes}


\question{Associate the correct definition to the following security mechanisms}
\begin{solution}
    - \textbf{Access Control}: Determines and enforce the access rights of the entity depending on the authenticated identityProvides confidentiality for either data or traffic flow information \\
    - \textbf{Encipherment}: Provides confidentiality for either data or traffic flow information \\
    - \textbf{Traffic Padding}: Protects against traffic analysis attacks by adding non essential data to network communications. Assures data integrity, origin, time, and destination about data communicated between two or more entities \\
    - \textbf{Notarization}: Assures data integrity, origin, time, and destination about data communicated between two or more entities \\
\end{solution}

\question{Associate the correct definition of the following Security Services:}
\begin{solution}
    - \textbf{Non-repudiation:} Assurance that someone cannot deny the validity of something. \\
    - \textbf{Confidentiality:} Ensuring that information is accessible only to those authorized to have access. \\
    - \textbf{Availability:} Ensuring that authorized users have access to information and associated assets when required. \\
    - \textbf{Integrity:} Maintaining and assuring the accuracy and completeness of data over its entire lifecycle. \\
\end{solution}

\question{What is the key difference between a monoalphabetic and a polyalphabetic cipher?}
\begin{checkboxes}
    \choice The method of rearranging the order of letters.
    \choice The use of multiple keys for encryption.
    \choice The length of the plaintext that can be encrypted.
    \CorrectChoice The complexity of the substitution pattern used.
    \choice None of the other options.
\end{checkboxes}

\question{Which of the following best defines a ciphertext-only attack in the context of wireless network security?}
\begin{checkboxes}
    \CorrectChoice An attack where the attacker can only access the ciphertext and attempts to decrypt it without any additional information.
    \choice An attack where the attacker has access to both the plaintext and its corresponding ciphertext.
    \choice An attack where the attacker manipulates the ciphertext to produce a predictable change in the plaintext.
    \choice An attack where the attacker intercepts and alters the communication between two parties.
    \choice None of the other options.
\end{checkboxes}
